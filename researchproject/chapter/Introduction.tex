\chapter{Einleitung}


\section{Motivation}
Sonnenlicht ist die wohl bedeutendste Energiequelle für uns Menschen. Durch Photosynthese gedeihen Pflanzen und bilden damit den Anfang der Nahrungskette. Wir gewinnen Strom durch Solarzellen, Wind- und Wasserkraftwerke, welche direkt oder indirekt durch Sonnenstrahlen ermöglicht werden.

Sonnenstrahlen spielen jedoch auch für unseres mentale Verfassung eine wichtige Rolle. Ohne sie kann der menschliche Körper kein Vitamin D-Erzeugen, was gesundheitliche Komplikationen mit sich ziehen kann. Zudem kann das Fehlen auf die Stimmung und die Motivation schlagen.

Gerade in den Wintermonaten können die kürzeren und schwächeren Sonnenperioden bei Menschen eine sogenannte \glqq Winterdepression\grqq{} verursachen \cite{Ban20}, die sich auf das persönliche wie professionelle Leben auswirken kann. Folglich ist es für jeden Betroffenen wünschenswert einem Lichtmangel entgegenzuwirken.

Das ausbleiben der Körpereigenen Vitamin-D Produktion kann durch die Einnahme von Nahrungsergänzungsmitteln ausgeglichen werden. Doch die ermunternde Wirkung von Tageslicht besitzen diese Mittel leider nicht.

\section{Zielsetzung}
Mit dem Aufkommen von LEDs, die das Strahlungsspektrum des sichtbaren Lichtes überzeugend simulieren können, haben sich sogenannte Tageslichtlampen auf dem Markt etabliert. Diese simulieren aber nur wenige Charakteristiken von echtem Sonnenlicht. Diese Arbeit soll die Idee eines Youtubers für ein Design einer Lampe mit realistischeren Charakteristika aufgreifen und in ein paar Aspekten verbessern.

Eine Eigenschaft der Beleuchtung mit Optimierungspotential ist die Stromaufnahme. Denn das Design der Lampe nutzt, je nach Konfiguration, eine 100w bis 500w starke LED. Diese ermöglicht eine realistisch helle Beleuchtung, hat aber eine entsprechend hohe Leistungsaufnahme. Um diese zu verbessern, sollte das Licht nur eingeschaltet sein, wenn nicht genügend natürliches Tageslicht zur Verfügung steht. 

Zudem sollte das Licht je nach Tageszeit an oder aus gehen, um nur in Wachspanne des Nutzers eingeschaltet zu sein. Eine solche Zeit-schalt-Funktion kann zudem dafür genutzt werden, den Nutzer zur richtigen Zeit mit \glqq Sonnenstrahlen\grqq{} zu wecken, oder daran erinnern, zeitnah ins Bett zu gehen.

\section{Vorgehensweise und Aufbau der Arbeit}

Der Aufbau dieser Arbeit ist im Folgenden beschrieben. Zunächst werden die physikalischen Grundlagen einer solchen Beleuchtungseinrichtung dargelegt. Danach wird erläutert, wie die genutzte Hardware- und Softwareplattformen gewählt wurden und ihre grundlegende Eignung für diesen Aufgabenbereich festgestellt wurde. Darauf aufbauend wird die Konzeption der Kommunikation zwischen den einzelnen Komponenten und die Gestaltung des User-Interfaces veranschaulicht. Im Anschluss wird die Implementierung und das Testvorgehen behandelt. Zuletzt wird das abschließende Ergebnis bewertet und ein Fazit aus dem resultierenden Ergebnissen gezogen.


