\chapter{Test}

\section{SunNode Test}
Um Erfahrungen mit Micropyhton zu sammeln und die zu verwendenden Konzepte aus zu testen, wurden schon vor der Entwicklungsphase atomare Tests durchgeführt. Dabei wurde die Netzwerkverbindung als Server sowie Client, die Messung des Lichtlevels über den Sensor, sowie die Ansteuerung der Relais ausgetestet. Später sollten Unit-Tests hinzugefügt werden. Dieser Plan wurde jedoch durch die fehlenden Standardmöglichkeiten zum Mocken von Micropython-Klasen stark erschwert. Auch nach langer Recherche wurde keine passende Alternative gefunden, weswegen auf manuelle Integrationstest gesetzt wurde. Zwar wäre eine eigene Implementation von Mock-Klassen grundsätzlich machbar, jedoch würde dies den Rahmen dieser Arbeit sprengen.
Bei dieser Art des Testens wurden  Speicherlimitation des Esps und oder architektonische Probleme erkenntlich. Denn durch die Speicherung von primitiven Zahlen wie Integers oder Floats als Objekte, benötigt Micropyhton ca. dreimal so viel Platz, als für den Wert alleine nötig wäre. Dies stellt eine sehr ineffiziente Art der Speicherung auf einer so stark limitierten Plattform da. Eine Möglichkeit dieses Problem zu umgehen wäre den Speicher an sich zu erweitern. Zu diesem Zweck gibt es Module für die Anbindung einer SD-Karte, die mit dem Esp kompatibel sind. Da für den vorgesehenen Einsatzzweck 15 Minutenintervalle als ausreichend angesehen wurden, konnte jedoch eine niedrigere Messfrequenz die Speicherlimitierung umgehen.

\section{SunControl Test}
SunControl hat einen sehr geringen Funktionsumfang. Da es fast ausschließlich für die Visualisierung von Daten der SunNode zuständig ist, macht es wenig Sinn viele Unit-Tests zu schreiben, da diese fast ausschließlich die Funktion der Getter- und Setter-Methoden des Android-Frameworks testen würden und ob sie in der richtigen Reihenfolge aufgerufen wurden. Eines Ausnahme bildet die TimeLabelFormatter-Klasse. Sie ist für die Umwandlung von geschriebener Zeit in eine Dezimalzahl und zurück zuständig.