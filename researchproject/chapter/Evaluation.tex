\chapter{Evaluation}
\label{SunNode}
Die SunNode erfüllt ihren vorgesehenen Zweck größtenteils. Das Licht schaltet sich bei den eingestellten Uhrzeiten und entsprechenden Beleuchtungsumständen an und auch wieder aus. Die Kommunikation mit SunControl verläuft zügig und gibt bei Veränderungen zeitnah Feedback. Dabei bleibt das Programm stabil und verhält sich vorhersehbar.
Es besteht jedoch ernsthaftes Verbesserungspotential, das über einen höheren Featureumfang hinausgeht. Denn durch die Verwendung von NTP schaltet der esp32 nicht automatisch zwischen Winter und Sommerzeit um. Dieser Umstand ist nicht durch Bordmittel von Mircopython zu ändern und benötigt eine eigene Implementierung. Da  Python diese Funktion mitbringt und die Nutzung von Zeit in Programmcodes nicht selten vorkommen sollte, stellt sich die Frage, wieso Micropython dies nicht anbietet. Mit dem momentanen Stand der Dinge hinkt die Uhr der SunNode während der Winterzeit eine Stunde zurück.

\section{SunControl}
Auch die SunControl erfüllt ihren vorgesehenen Zweck größtenteils. So zeigt sie relevante Informationen für den Nutzer an und reagiert zügig und vorhersehbar auf Eingaben. Das Nutzerinterface entspricht dabei dem ursprünglichem Konzept vollkommen. Bei der Anzeige kann jedoch ein Problem zu Tage treten: Die Anzeige wird nur einmal beim Öffnen der Applikation aktualisiert und nach jeder Interaktion mit den Einstellungsmöglichkeiten. Wenn sich das Licht während einer interaktionslosen Phase automatisch ausschaltet, oder ein anderer Nutzer Änderungen vornimmt, wird dies nicht in dem Nutzerinterface widergespiegelt. Dies liegt an der grundlegenden Kommunikationsweise zwischen SunControl und SunNode, die der SunNode nicht die Möglichkeit gibt, ihre Observer über Änderungen zu informieren. Dies klingt in der Theorie problematisch, sollte im alltäglichen Gebrauch jedoch kaum auffallen. Denn an einem typischen Tag sollte das Licht sich vier mal an- oder ausschalten, wodurch es nicht sehr wahrscheinlich ist, das dies bei einer vermutlich kurzen und seltenen Interaktion mit der Applikation, vorkommt. Auch Mehrnutzerbetrieb sollte so gut wie nicht vorkommen, da es sich um eine einzelne Beleuchtung handelt.

\section{Hardware}
Die Hardware erfüllt genau den Zweck ihres Konzepts. Das abgegebene Licht erinnert an Morgensonne und wurde von den bisherigen Nutzern als motivations- und konzentrations-steigernd empfunden.
Die Implementierung zeigt jedoch Schwachstellen am ursprünglichem Konzept. Denn wenn die Sonnen von Wolken verdeckt wird und das Lichtschaltlevel zwischen der Helligkeit bei freiem Himmel und verdecktem Himmel gesetzt wurde, kann es zum wiederkehrendem Ein- und Ausschalten des Lichtes kommen. 
Eine Modifikation des ursprünglichem Hardware-Designs könnte hier Abhilfe schaffen. So könnte ein Dimmer in Echtzeit die Helligkeit anpassen, ohne Flackern zu verursachen oder durch extreme Helligkeitswechsel zu irritieren.
