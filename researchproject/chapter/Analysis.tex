\chapter{Analyse}

\section{Hardware Analyse}

Für die vollständige Umsetzung der im letzten Abschnitt genannten Aspekte in einem Leuchtmittel sind die folgenden Hauptkomponenten nötig:

\begin{compactitem}
	\item Leuchtmittel mit genügend Ausgangsleistung
	\item Möglichkeit zum dimmen des Leuchtmittels
	\item Komponente zur Steuerung der Streucharakteristik des Lichts
	\item Komponente zur Erreichung einer realistischen Diffusion des Lichts
	\item Aufhängung, die ein automatisiertes Schwenken erlaubt
	\item Möglichkeit der Steuerung und Automatisierung der Beleuchtung
\end{compactitem}


\section{Funktionale Anforderungen}
Damit die Beleuchtung den Nutzer effektiv unterstützen kann, bedarf es einer Bedienoberfläche, mit der das gewünschte Verhalten der Lampe eingestellt werden kann. Da dieses Verhalten vom Nutzer abhängig ist und ihn bei seiner Tagesplanung unterstützen sollte, müssen mindestens die folgenden Parameter einstellbar und einsehbar sein:

\begin{compactitem}
	\item Automatische Ein- und Ausschaltzeit der Beleuchtung
	\item Schwellenwert, bei dem das Licht die Rolle der Sonne übernimmt
	\item Wechsel zwischen manuellem und autonomen Modus um das Licht wie eine reguläre Lampe verwenden zu können
\end{compactitem}

\section{Nichtfunktionale Anforderungen}
Die Nutzung der autonomen Funktion ist gerade für dieses Projekt Kernbereich der Arbeit. Im Idealfall sollte die Lampe nur ein einziges Mal eingestellt werden und danach vollkommen autonom ein- und ausschalten. Aus diesem Grund ist ein hohes Maß an Zuverlässigkeit für diese Funktion von Nöten. Des weiteren sollte das Nutzer-Interface möglichst intuitiv zu bedienen sein und die einzelnen Funktionen so wenig Interaktion wie möglich verlangen. Damit soll die reibungslose Nutzung sichergestellt werden und durch das Vertrauen auf den Autonomen Betrieb Energie gespart werden.