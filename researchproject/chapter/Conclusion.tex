\chapter{Fazit}

\section{Zusammenfassung}
Eine Beleuchtung zu designen, die Sonnenlicht realistisch Nachahmen kann, ist keine kleine Aufgabe. Es gibt viele Produkte, die Elemente davon simulieren können. Jedoch gibt es trotz verschiedener Konzepte bis jetzt keine Produkte auf dem Markt, das so viele Elemente davon vereint. Dies ist eine bemerkenswerte Marktlücke, den viele Bereiche des Lebens könnten von einem solchen Produkt profitieren. Nachtschichten könnten sich erfrischender anfühlen, Wintermonate sich nicht so lange anfühlen und sogar komplett neue Wohnkonzepte erschlossen werden.

\section{Kritischer Rückblick}
Diese Projektarbeit hat gezeigt, dass bereits alle Technologien für eine realistische Simulation von Sonnenlicht existieren. Es ist nur eine Frage der Implementierung, diese Stücke zu einem großen Ganzen zu vereinen. Ob tatsächlich jemand ein so helles Licht in seinem Wohnzimmer braucht, ist zu hinterfragen. Viele weitere Eigenschaften sind aber durchaus wünschenswert auch wenn nicht alle in diesem Projekt realisiert wurden. Der wortwörtlich größte Schwachpunkt dieser Implementierung ist die physikalischen Dimensionen der benötigen Hardware. Wenn man parallele ausgerichtetes Licht erzeugen möchte, braucht man eine Vorrichtung mit der selben Höhe und Breite, die der Strahl besitzen soll. In Verbindung mit einem Parabolspiegel kommt zudem eine proportional große Tiefe hinzu. Dieses Projekt verbraucht damit einen knappen Kubikmeter an Platz. Zudem ist für einen realistischen Effekt ratsam, das Licht wie ein (Dach)-Fenster zu platzieren. Dies stellt einen guten Grund für die geringe Adaption dieses Ansatzes in kommerziellen Produkten dar. 

\section{Ausblick}
Innovationen können die Limitationen dieser Lösung in Zukunft hoffentlich überkommen. Neue Beleuchtungskonzepte sind gefragter den je, was nicht zuletzt durch IKEAs hauseigene vernetzte Beleuchtungssystem erkenntlich ist. Bis sich kommerzielle Produkte weiterentwickelt haben, bleibt leider nur der Weg des Eigenbaus.