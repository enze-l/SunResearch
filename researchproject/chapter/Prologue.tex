\chapter*{Vorwort}

Die vorliegende Projektarbeit entstand im Rahmen meines Studiums der Angewandten Informatik an der Hochschule für Technik und Wirtschaft Berlin.

Die Idee zu diesem Projekt entstand am Anfang der Wintermonate, als das Tageslicht immer weniger wurde. Wie vermutlich bei vielen weiteren Menschen, fällt bei mir die Stimmung, und damit auch die Produktivität, bei zu wenig Tageslicht ab. Um gegen diesen Umstand vorzugehen, suchte ich nach Lösungen für dieses Problem. Dabei stieß ich auf ein Design einer Beleuchtung, die nicht nur die Farbtemperatur von Sonnenlicht, sondern auch die Parallelität der Strahlen, die diffuse Streuchcharakteristiken und sogar ansatzweise die Intensität nachahmt. Eine der Kernkomponenten dieser Lampe ist ein 100w-500w starke LED.

Schon vor Beginn des Nachbaus dieser Idee, kamen mir einige Bedenken. Zum einem verbraucht eine solche Beleuchtung ein vielfaches des Stroms konventioneller und zeitgemäßer Leuchtmittel. Zum anderen sah ich die Gefahr durch die Beleuchtung meinen Tagesrhythmus unbeabsichtigt aus dem Rhythmus zu bringen.

Um sicherzustellen, dass diese Beleuchtung nur an ist, wenn sie aufgrund mangelndem natürlichem Lichts benötigt wird und sie außerdem den Nutzer nicht aus versehen bis tief in die Nacht wach hält, bot sich eine Automatisierung der Schaltung an. Diese Arbeit soll den Prozess der Realisierung der Hard- sowie Software darstellen.

Mein Dank gilt Prof. Dr. Huhn für die Beratung zur Wahl der elektronischen Komponenten sowie der Software-Plattform und meiner Freundin Juliya Rajasingam für die Hilfe bei der Beschaffung aller nicht elektronischen Komponenten und als (mehr oder weniger) freiwilliges Testsubjekt.

Ich hoffe allen Lesern gut die Erfahrungen vermitteln zu können, die ich bei der Durchführung dieses Projektes gesammelt habe.


Leon Enzenberger

Berlin, 24.06.2020
