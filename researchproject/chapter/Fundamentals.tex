\chapter{Pysikalische Grundlagen}

Sonnenlicht hat auf der Erdoberfläche einige besondere Charakteristiken, die sich im Laufe des Tages auch verändern können. Dazu zählen unter anderem:

\begin{compactitem}
	\item Farbtemperatur
	\item Aufteilung in diffusen und direkten Anteil
	\item Einfallswinkel
	\item Parallelität der Strahlen
	\item Intensität
\end{compactitem}

Die \textbf{Farbtemperatur} des Sonnenlichtes kommt von der spezifischen chemischen Zusammensetzung der verschiedenen strahlenden Elemente. 

Wenn sich dieses Licht durch die Atmosphäre der Erde bewegt, werden die kurzwelligen Strahlen durch Partikel und Moleküle einfacher abgelenkt als Langwellige. So erscheint die Sonne mittags gelb (langwelligere Strahlen), im Gegensatz zum blauen Himmel (kurzwelligere Strahlen), der durch das abgelenkte Licht erleuchtet ist. Diese Aufteilung führt zu einer \textbf{diffus} blauen und einer \textbf{direkten} gelben Beleuchtung.

Durch die Drehung der Erde verändert sich der \textbf{Einfallswinkel} der Sonnenstrahlen jedoch. So müssen sie morgens und abends mehr Atmosphäre durchdringen als Mittags. Hierdurch kann noch mehr Licht gefiltert werden, wodurch sich die \textbf{Farbtemperatur} und \textbf{Intensität} des Lichtes stark ändern kann. Den farblichen Unterschied kann mit Kelvin gemessen werden, die Intensität mit Lux.

Die nahezu \textbf{parallele Orientierung der Lichtstrahlen} kommt von der weiten Entfernung der Sonne zur Erde. Der Effekt ist, dass ein Objekt immer einen gleich großen Schatten wirft, egal wie weit es von der Lichtquelle entfernt ist.
